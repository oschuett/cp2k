%\documentclass[paper=A4, BCOR=8.00mm, draft=false, DIV=13, parskip=half, headsepline=true]{scrbook}
\documentclass{article}

\usepackage[utf8]{inputenc}
\usepackage[english]{babel}
\usepackage{amsmath,amssymb}
\usepackage{bbm}
\newcommand{\tr}{\operatorname{tr}}

\setlength{\parskip}{2ex}
\setlength{\parindent}{0pt}

\begin{document}



%==============================================================================
\section*{PAO LNV for arbitrary A and B matrices}

The LNV energy functional is:

\begin{align*}
    \Omega & = 3\tr[\tilde P \tilde S \tilde P \tilde H ] - 2\tr[\tilde P \tilde S \tilde P \tilde S \tilde P \tilde H]
\end{align*}

The tilded variables are in the smaller PAO-basis. The transformations between the larger primary basis and the PAO-basis are:
\begin{align*}
    \tilde P &= A^T  P A &
    \tilde H &= B^T  H B &
    \tilde S &= B^T  S B
\end{align*}

Notice that: $ A^T B = B^T A = \tilde I $, where $\tilde I$ is the identity matrix in the smaller PAO basis.

The relationship between $A$ and $B$ is defined via the Moore-Penrose pseudo-inverse:
\begin{align*}
    B & = A (A^T A) ^{-1}
\end{align*}



%==============================================================================
\newpage
\section*{1. Derivative}

\begin{align*}
    \frac{d \Omega}{d A_{ij}}
 &= \sum_{ab} \frac{\partial \Omega}{\partial \tilde P_{ab}}  \frac{\partial \tilde P_{ab}}{\partial A_{ij}}
  + \sum_{ab} \frac{\partial \Omega}{\partial \tilde H_{ab}}  \frac{\partial \tilde H_{ab}}{\partial A_{ij}}
  + \sum_{ab} \frac{\partial \Omega}{\partial \tilde S_{ab}}  \frac{\partial \tilde S_{ab}}{\partial A_{ij}}\\
  %
 &= \sum_{ab} M^1_{ab} \frac{\partial \tilde P_{ab}}{\partial A_{ij}}
  + \sum_{ab} M^2_{ab} \frac{\partial \tilde H_{ab}}{\partial A_{ij}}
  + \sum_{ab} M^3_{ab} \frac{\partial \tilde S_{ab}}{\partial A_{ij}}\\
  %
 &= T^1_{ij} + T^2_{ij} + T^3_{ij}
 %&= 2 N^{-1T} P A M^1 Y^T
 % + 2 N^{T}   H B M^2 Y^T
 % + 2 N^{T}   S B M^3 Y^T
\end{align*}

\begin{align*}
T^1_{ij} &= \sum_{ab} M^1_{ab} \frac{\partial \tilde P_{ab}}{\partial A_{ij}} \\
         &= \sum_{ablk} M^1_{ab} \frac{\partial}{\partial A_{ij}} A_{la} P_{lk}  A_{kb} \\
         &= \sum_{ablk} M^1_{ab} \left [ \delta_{li} \delta_{aj} P_{lk}  A_{kb} + A_{la} P_{lk} \delta_{ki} \delta_{bj} \right ]\\
         &= \left [ P A M^{1T} \right ]_{ij}  +  \left [ P^T A M^1 \right ]_{ij} \\
         &= 2 \left [ P A M^{1} \right ]_{ij}
\end{align*}

\begin{align*}
T^2_{ij} &= \sum_{ab} R^2_{ab} \frac{\partial B_{ab}}{\partial A_{ij}}
\end{align*}

\begin{align*}
T^3_{ij} &= \sum_{ab} R^3_{ab} \frac{\partial B_{ab}}{\partial A_{ij}}
\end{align*}

\newpage

\begin{align*}
R^2_{ij} &= \sum_{ab} M^2_{ab} \frac{\partial \tilde H_{ab}}{\partial B_{ij}} \\
         &= \sum_{ablk} M^2_{ab} \frac{\partial}{\partial B_{ij}} B_{la} H_{lk} B_{kb} \\
         &= \sum_{ablk} M^2_{ab} \left [ \delta_{li} \delta_{aj} H_{lk} B_{kb}  +  \delta_{ki} \delta_{bj} B_{la} H_{lk} \right] \\
         &= \left [H B M^{2T} \right]_{ij} + \left [ H^{T}B M^2 \right ]_{ij}  \\
         &= 2 \left [H B M^{2} \right]_{ij}
\end{align*}

Educated guess:
\begin{align*}
R^3_{ij} &= \sum_{ab} M^3_{ab} \frac{\partial \tilde S_{ab}}{\partial B_{ij}}\\
         &= 2 \left [S B M^{3} \right]_{ij}
\end{align*}

\newpage

\begin{align*}
    M^1_{ab} &= \frac{\partial \Omega}{\partial \tilde P_{ab}}\\
    & = \frac{\partial}{\partial \tilde P_{ab}} \left (3\tr[\tilde P \tilde S \tilde P \tilde H ] - 2\tr[\tilde P \tilde S \tilde P \tilde S \tilde P \tilde H] \right )\\
    & = 3 \tilde H \tilde P \tilde S  + 3 \tilde S \tilde  P \tilde H        -2 \tilde H\tilde P\tilde S\tilde P\tilde S   -2 \tilde S\tilde P\tilde H\tilde P\tilde S  -2\tilde  S\tilde P\tilde S\tilde P\tilde H
\end{align*}

\begin{align*}
    M^2_{ab} &= \frac{\partial \Omega}{\partial \tilde H_{ab}}\\
    & = \frac{\partial}{\partial \tilde H_{ab}} \left (3\tr[\tilde P \tilde S \tilde P \tilde H ] - 2\tr[\tilde P \tilde S \tilde P \tilde S \tilde P \tilde H] \right )\\
    & = 3 \tilde P \tilde S \tilde P - 2 \tilde P \tilde S \tilde P \tilde S \tilde P
\end{align*}

\begin{align*}
    M^3_{ab} &= \frac{\partial \Omega}{\partial \tilde S_{ab}}\\
    & = \frac{\partial}{\partial \tilde S_{ab}} \left (3\tr[\tilde P \tilde S \tilde P \tilde H ] - 2\tr[\tilde P \tilde S \tilde P \tilde S \tilde P \tilde H] \right )\\
    & =  3\tilde P \tilde H  \tilde P - 2 \tilde P \tilde H\tilde P\tilde  S\tilde  P  - 2 \tilde P\tilde S \tilde P\tilde H\tilde P
\end{align*}


%==============================================================================
\end{document}

%EOF
